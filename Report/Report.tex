\documentclass[11pt,a4paper]{article}
\usepackage{etex}
\reserveinserts{28}

\usepackage[utf8]{inputenc}

\usepackage{hyperref}
\usepackage{amsthm}
\usepackage{amsmath}
\usepackage{amssymb}
\usepackage{amsfonts}
\usepackage[portuguese]{babel}
\usepackage{pictex}
\usepackage[dvips]{graphics}
\usepackage{enumerate}
\usepackage{color}
\usepackage[usenames,dvipsnames]{xcolor}

\usepackage{amsmath}
\usepackage{amssymb}
\usepackage{amsfonts}
\usepackage[portuguese]{babel}
\usepackage{pictex}
\usepackage[dvips]{graphics}
\usepackage{enumerate}
\usepackage{color}
\usepackage{indentfirst}
\usepackage{listings}
\usepackage{tabularx}
\usepackage{multirow}
\usepackage{float}
\usepackage{makeidx}

\usepackage{graphicx}
\setlength{\parindent}{1cm}
\title{\bf{Computação Gráfica}\vspace{50mm}\\Projecto\vspace{80mm}}
\author{
João Miguel Moreira de Carvalho Brás Simões, Nº 2011150045\\
João Ricardo Maximiano Leitão Ribeiro Lourenço, Nº 2011151194\\
Joaquim Pedro Bento Gonçalves Pratas Leitão, Nº 2011150072}
\makeindex
\begin{document}
\maketitle
\centerline{\textbf{Relatório}}
\pagebreak

\printindex

\pagebreak

\section{Introdução}

Com o presente trabalho pretende-se aprofundar e aplicar os conhecimentos abordados \emph{Computação Gráfica}, trabalhando outros aspetos não lecionados na disciplina.

Os principais requisitos e componentes do projeto consistem na criação de uma aplicação gráfica utilização de \emph{OpenGL} e na presença de um modelo de um robot, elemento chave do 

\end{document}
